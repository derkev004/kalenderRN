\documentclass{minimal}

\usepackage{forloop,calc,ifthen,intcalc}

\newcommand{\emptytype}{\typeout{}\typeout{}\typeout{}}

\newcommand{\myend}{\begin{document}\end{document}}

\let\equals\equal

\newcommand{\geteom}[1]
{
\ifcase#1
  \or 31 %1 Jan
  \or \ifnum\intcalcMod{\year}{4}=0 29 \else 28 \fi %2 Feb (evtl. Schaltjahr)
  \or 31 %3 Mär
  \or 30 %4 Apr
  \or 31 %5 Mai
  \or 30 %6 Jun
  \or 31 %7 Jul
  \or 31 %8 Aug
  \or 30 %9 Sep
  \or 31 %10 Okt
  \or 30 %11 Nov
  \or 31 %12 Dez
\fi
}

\newcommand{\getdayofweek}[3]
{%
\intcalcMod{%
\ifnum#2<3 #1-1\else#1\fi +%
\intcalcDiv{\ifnum#2<3 #1-1 \else#1\fi}{4} -%
\intcalcDiv{\ifnum#2<3 #1-1 \else#1\fi}{100} +%
\intcalcDiv{\ifnum#2<3 #1-1 \else#1\fi}{400} +%
\ifcase#2\or%
0\or%1
3\or%2
2\or%3
5\or%4
0\or%5
3\or%6
5\or%7
1\or%8
4\or%9
6\or%10
2\or%11
4\else%12
0\fi%
+ #3 + 6}{7}%
}

\xdef\bon@year{2018}

\newcount\myhours
\newcount\myminutes

\myhours   = \time
\divide   \myhours by 60

\myminutes = \time
\multiply \myhours by 60

\advance  \myminutes by -\myhours
\divide   \myhours by 60

\def\twochars#1{\ifnum#1<10 0#1\else#1\fi}

\def\timestamp{\the\year\twochars{\the\month}\twochars{\the\day}T\twochars{\the\myhours}\twochars{\the\myminutes}00Z}
\def\uid{\timestamp-\bon@year\twochars{\arabic{monatcnt}}\twochars{\arabic{tagcnt}}\twochars{\arabic{schichtcnt}}}

\newcommand{\iCalPreamble}{%
\emptytype
\typeout{BEGIN:VCALENDAR}
\typeout{METHOD:PUBLISH}
\typeout{VERSION:2.0}
\typeout{CALSCALE:GREGORIAN}
\typeout{PRODID:Kevin F. Konrad//kfkonrad//gen_iCal.sh}
}

\newcommand{\iCalEvent}[2]{%
\typeout{BEGIN:VEVENT}
%
%\typeout{DTSTART:#1T000000Z}
%\typeout{DTEND:#1T000100Z}
\typeout{UID:\uid}
\typeout{DTSTAMP:\timestamp}
\typeout{DTSTART;VALUE=DATE:#1}
\typeout{SUMMARY:#2}
\typeout{TRANSP:TRANSPARENT}
%
\typeout{BEGIN:VALARM}
\typeout{TRIGGER;VALUE=DATE-TIME:19760401T005545Z}
\typeout{ACTION:NONE}
\typeout{END:VALARM}
%
\typeout{END:VEVENT}
}

\newcommand{\iCalPostamble}{%
\typeout{END:VCALENDAR}
\emptytype
}

\newcommand{\maketwochars}[2]{
\ifthenelse{#1<10}
{\def\monthnum{0#1}}
{\def\monthnum{#1}}
\ifthenelse{#2<10}
{\def\daynum{0#2}}
{\def\daynum{#2}}
}

%\iCalPreamble
%\iCalEvent{20170205}{Testfall}
%\iCalEvent{20170206}{Testfall}
%\iCalPostamble
%\myend

\iCalPreamble

\newcounter{monatcnt}
\setcounter{monatcnt}{1}
\newcounter{tagcnt}
\setcounter{tagcnt}{1}
\newcounter{wochentagcnt}
\setcounter{wochentagcnt}{\getdayofweek{\bon@year}{1}{1}}
\newcounter{schichtcnt}
\setcounter{schichtcnt}{3} %Schichtanfang ist letzter freier Tag

\forloop{monatcnt}{1}{\arabic{monatcnt} < 13}
{

  \forloop{tagcnt}{1}{\arabic{tagcnt} < \numexpr \geteom{\arabic{monatcnt}} + 1 \relax}
  {
  \stepcounter{wochentagcnt}
  \setcounter{wochentagcnt}{\intcalcMod{\arabic{wochentagcnt}}{7}}
  \ifthenelse{ \arabic{wochentagcnt}=4 }
  {}
  {
    %\typeout{\intcalcMod{\arabic{wochentagcnt}}{7}}
    \stepcounter{schichtcnt}
    \setcounter{schichtcnt}{\intcalcMod{\arabic{schichtcnt}}{5}}
  }

  \ifthenelse{\arabic{wochentagcnt} = 0}
  {
    \ifthenelse{\arabic{schichtcnt} = 2}
    {\def\thistext{Früh}}
    {
      \ifthenelse{\arabic{schichtcnt} = 3}
      {\def\thistext{Nacht}}
      {\def\thistext{frei}}
    }
  }
  {
    \ifthenelse{ \arabic{schichtcnt} = 1}
    {\def\thistext{früh}}
    {
      \ifthenelse{ \arabic{schichtcnt} = 2}
      {\def\thistext{spät}}
      {
        \ifthenelse{ \arabic{schichtcnt} = 3}
        {\def\thistext{nacht}}
        {\def\thistext{frei}}
      }
    }
  }

  \maketwochars{\arabic{monatcnt}}{\arabic{tagcnt}}
  %\typeout{\daynum}
  %\typeout{\monthnum}
  %\iCalEvent{20170205}{Testfall}
  \ifthenelse{ \equal{\thistext}{frei}}
  {}
  {\iCalEvent{\bon@year\monthnum\daynum}{\thistext}}

  }
}

\iCalPostamble

\myend
